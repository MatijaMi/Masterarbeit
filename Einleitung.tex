\chapter{Introduction}
The topic of this thesis the dynamic insertion of 3D objects, defined in CAD Files, into an Unreal Engine program while it is running. Especially important for the project are why this might even a problem and how it can actually be realized. For these purposes an Unreal Engine plug-in was developed, which enables such a feature, and an additional Unreal Engine program which implements the plug-in and demonstrates some ways of interacting with objects generated in such a way.

%%%%%%%%%%%%%%%%%%%%%%%%%%%%%%%%%%%%%%%%%%%%%%%%%%%%%%%%%%%%%%%%%%%%%%%%%%%%%%%%%%%
\section{Motivation}


%%%%%%%%%%%%%%%%%%%%%%%%%%%%%%%%%%%%%%%%%%%%%%%%%%%%%%%%%%%%%%%%%%%%%%%%%%%%%%%%%%%
\section{Goals}\label{chp:Goals}   
The main goal of this work is to develop an efficient and user-friendly plug-in which will make it possible to load 3D objects from the most common CAD formats during the runtime of an Unreal Engine program. Additionally another software should be developed to use the plug-in and allow simple interactions with these objects in a normal desktop window as well as in a virtual reality environment.

%%%%%%%%%%%%%
\subsubsection{Efficiency} 
The developed plug-in should be capable of handling large amounts of data seeing as the models which can be found in CAD files can be incredibly large, containing thousands or millions of vertices and polygons. If the plug-in were to effect the runtime performance in a significant way, such as causing stutters or freezing the program all together, it would severely worsen the user experience and invalidate the whole point of the program.

%%%%%%%%%%%%%
\subsubsection{Expandability} 
The field of computer assisted design is very wide and there are countless programs and formats for all the varying use-cases in which it is being used. That is why creating one solution for all of those is incredibly complicated and way out of the scope and possibilities of this project. Instead it is much better to concentrate on creating a simple to use and understand system which can then be further improved upon and adjusted for the concrete cases of clients or projects.

%%%%%%%%%%%%%%%%%%%%%%%%%%%%%%%%%%%%%%%%%%%%%%%%%%%%%%%%%%%%%%%%%%%%%%%%%%%%%%%%%%%
\section{Thesis Structure}
In Chapter \ref{chp:UnrealEngine} the Unreal Engine will be clarified and explained. Seeing as this is both the tool which is being used for development as well as being the software for which the plug-in is being developed, an understanding of how it works and what its limitations are is needed in order to better grasp the project and what problems might arise. It is a rather expansive tool so not everything will be covered, only the more basic aspects and the concrete parts which play a role for this project. Then, in Chapter \ref{chp:ObjectLoading}, the plug-in will be analysed, starting of with how the files are parsed and into what sort of form they are transformed in order to be used. After that comes the actual mesh generation mesh, how it is achieved and where  extra attention is required. In Chapter\ref{chp:ObjectInteraction} it will be illustrated in what ways users can interact with the newly created objects, either using mouse and keyboard or a virtual reality headset. In Chapter \ref{chp:Results} the developed programs will be presented, evaluated and compared to similar software to see where its strengths and weaknesses are. Lastly in Chapter \ref{chp:Conclusion} the reached goals and some possible further projects and improvements will be discussed.


%%%%%%%%%%%%%%%%%%%%%%%%%%%%%%%%%%%%%%%%%%%%%%%%%%%%%%%%%%%%%%%%%%%%%%%%%%%%%%%%%%%
\section{Related Works}
 
