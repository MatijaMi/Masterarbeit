% das Papierformat zuerst
%
%\documentclass[a4paper, 11pt,bibtotoc]{scrartcl}
\documentclass[a4paper, 11pt,bibtotoc,abstracton]{scrreprt}  

%Für URL
\usepackage{url}
\renewcommand{\UrlFont}{\rmfamily}

%Für Zitaten
\usepackage{cite}

%Abkürzungen
\usepackage{acronym}


%Inhaltsverzeichnis bearbeiten
\usepackage{tocbibind}

% für mathematische Symbole
\usepackage{amsmath}


% Vektorgrafiken mit Latex importieren
\usepackage{import}

\usepackage[colorlinks=false, pdfborder={0 0 0}]{hyperref}
%documentclass[a4paper, 12pt]{article}  
% deutsche Silbentrennung
\usepackage[english]{babel}
\renewcommand{\sectfont}{\rmfamily\bfseries}
% wegen deutschen Umlauten
\usepackage[utf8]{inputenc}
\usepackage{graphicx}
\usepackage{subfigure}\hyphenation{Bit-rate}

%Algorithm schreiben
\usepackage{algorithm2e}

%Tabellen
\usepackage{booktabs}
\usepackage{multirow}
\usepackage{colortbl}
\usepackage{wrapfig}

%Farben
\usepackage{color}
\usepackage{listings}%Code einbinden
\definecolor{darkblue}{rgb}{.08,.21,.36}
\definecolor{darkred}{rgb}{.6,.19,.20}
\definecolor{darkgreen}{rgb}{0,.6,0}
\definecolor{red}{rgb}{.98,0,0}
\definecolor{lightblue}{rgb}{0.8,0.85,1}
%\definecolor{lightgrey}{rgb}{0.98,0.98,0.98}
\definecolor{lightgrey}{gray}{.98}
\definecolor{black}{rgb}{0.0,0.0,0.0}


\lstloadlanguages{C++}
\lstset{%
  language=C++,
  basicstyle=\small,
  commentstyle=\itshape\color{darkgreen},
  keywordstyle=\bfseries\color{darkblue},
  stringstyle=\color{darkred},
  showspaces=false,
  showtabs=false,
  columns=fixed,
  backgroundcolor=\color{lightgrey},
  numbers=left,
  frame=single,
  numberstyle=\tiny,
  breaklines=true,
  showstringspaces=false,
  xleftmargin=1cm,
  basicstyle=\small
}%

\usepackage{amssymb}%Mathematische Symbole, wie R,N,Q,Z,...
\setlength{\parindent}{0pt} %einrücken nach absatz verhindern
%\usepackage{setspace}%Zeilenabstand
\usepackage{algorithmic}%Für Pseudocode

%%%%%%%%%%%%%%%%%%%%%%%%%%%%%%%%%%%
%Seiten Kopf- und Fußzeilen
\usepackage[automark,						
		headsepline,								
		plainfootsepline, 
		]{scrlayer-scrpage}

\automark[section]{chapter} 
\pagestyle{scrheadings}			

\clearscrheadings	%Alte Kopfformatierungen entfernen
\clearscrplain		%Alte Plain-Formatierung entfernen
\clearscrheadfoot %Alten Fuß entfernen
\cfoot[\pagemark]{\pagemark}%Seitenzahl zentriert im Fuß 
\ihead{\leftmark}
\ohead{\rightmark} 
 
%%%%%%%%%%%%%%%%%%%%%%%%%%%%%%%%%%%

\begin{document}
\begin{titlepage}
\begin{center}
{\huge \textbf{Philipps-Universität Marburg}}\\[0.5cm]
\textbf{Fachbereich 12 - Mathematik und Informatik}\\[0.5cm]

\begin{figure}[h]
	\centering
		\includegraphics[width=0.8\textwidth]{fig/unilogo.pdf}
\end{figure}

{\huge \textbf{{\large \\[1cm]Master Thesis}}}
\\[1cm]

{\Huge \textbf{Dynamic Insertion of 3D Objects\\ from CAD Files into Unreal Engine}}
\\[1cm]

{\large  Matija Mišković}\\
{\large September 2022}\\[3cm]

{\large
Supervisor:\\ Prof. Dr. Thorsten Thormählen\\[1cm]
Research Group Graphics and Multimedia Programming}

\end{center}
\end{titlepage}
\newpage
\thispagestyle{empty}
\section*{}
\newpage
\thispagestyle{empty}
\vspace*{7cm}
\textbf{\Large {Declaration of Originality}}\\[0.5cm]
I,  Matija Mišković (Computer Science Student at Philipps-University Marburg, Student-ID:
3139015), confirm that the submitted thesis is original work and was written by me without further assistance. Appropriate credit has been given where reference has been made to the work of others.
The thesis was not examined before, nor has it been published. The submitted electronic version of
the thesis matches the printed version.\\[1cm]
Marburg, 29. September 2022\\[0.5cm]
 Matija Mišković
\newpage
\shipout\null

%\setstretch {1.15}%Zeilenabstand setzen

  
% hier beginnt das Dokument