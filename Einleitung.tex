\chapter{Introduction}
The topic of this thesis the dynamic insertion of 3D objects, defined in computer assisted design(CAD) Files, into an Unreal Engine program while it is running. Especially important for the project are why this might even a problem and how it can actually be realized. For these purposes an Unreal Engine plug-in was developed which enables such functionality and an additional Unreal Engine program which implements the plug-in and can be used to present and interact with the objects in a multi-user desktop or virtual reality environment.

%%%%%%%%%%%%%%%%%%%%%%%%%%%%%%%%%%%%%%%%%%%%%%%%%%%%%%%%%%%%%%%%%%%%%%%%%%%%%%%%%%%
\section{Motivation}

Virtual reality(VR) is a relatively new field which is constantly seeing a lot of interest and innovation for all the new possibilities in opens up in software development and user interaction. In recent years VR has been used in many companies in various industries such as the engineering, architecture and healthcare and this number keeps on growing. One such company is Inosoft\cite{}.\\
Inosoft is a software development firm in Marburg which was founded in 1993 and has since worked and consulted over two thousand projects for various companies including Viessmann, CSL Behring, Sanofi and many more. They are also very interested in VR and have been working in the field since 2016. Inosoft was also interested in establishing a working relationship with the Phillips University Marburg. As such they reached out with some very interesting projects in the field of VR. Among them was designing and developing a concept to dynamically insert and interact with objects from CAD files in a running Unreal Engine environment.\\
There are definitely certain scenarios where this could be a very useful tools. As an example, let's take a software Inosoft developed which is used to train workers in a digital production plan while the physical building was being built. This quite a handy tool and has helped quite a bit\cite{Inosoft}. Slight problems arise when things about the model need to be added or changed. First the changes need to be implemented in Unreal, packaged for standalone use and then redistributed to everyone who needs to use them. It would be a lot simpler if the program could simply open a file and add the new or update objects without ever having to change the version of it.\\
Another use-case where this could be useful is in collaborative design or presenting 3D models. Instead of having to make the scene and import everything beforehand and distribute this version of the program, simply having a program that can open a file and have the model appear for everyone involved could save a lot of time and effort.\\
So seeing as there are uses for this technology it makes sense to look into how it could be done and what the limitations are, as well as looking into why this isn't already officially part of Unreal Engine.

%%%%%%%%%%%%%%%%%%%%%%%%%%%%%%%%%%%%%%%%%%%%%%%%%%%%%%%%%%%%%%%%%%%%%%%%%%%%%%%%%%%
\section{Goals}\label{chp:Goals}   
The main goal of this work is to develop an efficient and user-friendly plug-in which will make it possible to load 3D objects from the most common CAD formats during the runtime of an Unreal Engine program. Additionally another software will be developed to use the plug-in and allow simple interactions with these objects in a normal desktop window as well as in a virtual reality environment.

%%%%%%%%%%%%%
\subsubsection{Efficiency} 
The developed plug-in should be capable of handling large amounts of data seeing as the models which can be found in CAD files can be incredibly large, containing thousands or millions of vertices and polygons. If the plug-in were to effect the runtime performance in a significant way, such as causing stutters or freezing the program all together, it would severely worsen the user experience and invalidate the whole point of the program.

%%%%%%%%%%%%%
\subsubsection{Expandability} 
The field of computer assisted design is very wide and there are countless programs and formats for all the varying use-cases in which it is being used. That is why creating one solution for all of those is incredibly complicated and way out of the scope and possibilities of this project. Instead it is much better to concentrate on creating a simple to use and understand system which can then be further improved upon and adjusted for the concrete cases of clients or projects.

%%%%%%%%%%%%%%%%%%%%%%%%%%%%%%%%%%%%%%%%%%%%%%%%%%%%%%%%%%%%%%%%%%%%%%%%%%%%%%%%%%%
\section{Thesis Structure}
In Chapter \ref{chp:UnrealEngine} the Unreal Engine will be clarified and explained. Seeing as this is both the tool which is being used for development as well as being the software for which the plug-in is being developed, an understanding of how it works and what its limitations are is needed in order to better grasp the project and what problems might arise. It is a rather expansive tool so not everything will be covered, only the more basic aspects and the concrete parts which play a role for this project. Then, in Chapter \ref{chp:ObjectLoading}, the plug-in will be analysed, starting of with how the files are parsed and into what sort of form they are transformed in order to be used. After that comes the actual mesh generation mesh, how it is achieved and where  extra attention is required. In Chapter \ref{chp:ObjectInteraction} it will be illustrated in what ways users can interact with the newly created objects, either using mouse and keyboard or a virtual reality headset. In Chapter \ref{chp:Results} the developed programs will be presented, evaluated and compared to similar software to see where its strengths and weaknesses are. Lastly in Chapter \ref{chp:Conclusion} the reached goals and some possible further projects and improvements will be discussed.


%%%%%%%%%%%%%%%%%%%%%%%%%%%%%%%%%%%%%%%%%%%%%%%%%%%%%%%%%%%%%%%%%%%%%%%%%%%%%%%%%%%
\section{Related Works}

When it comes to this topic there are unfortunately not that many similar works. When it comes generally importing CAD files into Unreal Engine, Datasmith definitely needs to be mentioned.\\ Datasmith is an official set of tools and plug-ins created by Epic Games, the developers of Unreal Engine, to simplify and streamline the process of importing various CAD formats into the engine\cite{}. It is important to note that the main focus of Datasmith is to make the process of transferring a model from a CAD software into the Unreal Engine editor during development smoother and more efficient\cite{}. Nonetheless amongst the many features it has it does also contain a plug-in for loading the models in runtime. This plug-in is still in being developed, even upon installation there are clear warnings that the software is still in a beta, so there are some missing features and there also haven't seemingly been many updates to it since the initial research for this project started\cite{}.\\
Outside of Datasmith there are a handful of small plug-ins that can be found which handle this topic, most importantly glTFRuntime\cite{} and Runtime FBX Import\cite{}. They were developed by a small team and a single person respectively and are available to be bought in the Unreal Marketplace. Runtime FBX Import is decent but quite a few bugs were found while experimenting with it. In contrast, gltfRuntime works a lot better and is even comparable to Unreal's Datasmith. That's why the strengths and weaknesses of glTFRuntime, as well as Datasmith, will be discussed in further detail later in order better evaluate the programs developed for this project. 