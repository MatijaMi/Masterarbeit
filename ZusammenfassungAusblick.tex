\chapter{Conclusion}\label{chp:Conclusion}

Unreal Engine is an incredibly interesting and very unique piece of software with all of its features and functionalities. It is no surprise that it is seeing so much interest from various industries outside of game development. It is also no surprise that Unreal has to adapt in order to better suit these brand-new use-cases which bring with them completely new demands.\\
One example of that, the dynamic insertion of 3D objects from external CAD files, was demonstrated in this project. It is something that in traditional game development would make little to no sense. Considering the results of the performance test seen in Chapter \ref{chp:Results}, it is quite clear why this is the case. Nevertheless, it also can not be denied that there is a certain demand for such a feature and that Unreal does have the means for it in its system. The only real problem is somehow getting access to it.\\
One way to realize this was used for the development of the \acs{CRI} plug-in. While it is slightly unfortunate that a separate plug-in had to be used, Runtime Mesh Component is brilliant and without it the achieved results probably would not have been possible in the limited amount of time available for this project. Notably one of the main goals of the project, which is efficiency. At its current state, when it comes purely to generating vertices and faces, the performance is even better than expected, being comparable to the official solution from Epic Games. Of course, it needs to be repeated that Datasmith Runtime is still in beta and that it is probably going to get better as development continues. It would especially be interesting to see just how close the performance of runtime generated objects can get to normally generated ones. It is feasible that some amount of changes to the low-level systems of Unreal could be necessary to achieve that.\\
The other goal of the plug-in, expandability, was also achieved. With the documentation in the repository and in the source code, it should be quite simple to adjust and widen the capabilities of it. Unfortunately, this is undeniably needed as there are still some limitations to functionalities caused by the lack of time that was available to be spent on them.\\
The area where runtime generation will probably see the most adoption is in projects like CAD Runtime Presenter. Being able to quickly load in a new model and simply present it to many users in VR can be quite handy. \acs{CRP} also already showed some of the things that can be done with interaction, but that is without a doubt just a small fraction of the possibilities. It is conceivable that not only presentation but also designing and creating new models could be done during runtime with multiple users. Creating a selection of simpler shapes that can be adjusted and attached to each other in different ways should definitely be possible. With dynamic materials instances, the materials could easily be changed and adapted as that only requires changing some values. This does raise a question in the opposite direction of this project. Would it be possible to generate completely new models in runtime and then somehow save them to actual CAD formats? Would perhaps a new format be necessary? Also, for models that were loaded in from external files, would there be a way to save changes made to it in runtime directly to the original file?\\
All in all, this a very interesting topic and there are definitely countless more projects, both official and unofficial, that will be made to further expand what is possible with Unreal Engine. Maybe it will even come full circle and some creative developer will find a use for it in game development.
